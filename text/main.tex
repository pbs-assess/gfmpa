% !TEX TS-program = pdflatexmk

\documentclass[12pt]{article}
\usepackage[utf8]{inputenc}
\usepackage[margin=1.15in]{geometry}
\usepackage{amsmath, amsfonts, amssymb}
\usepackage[dvipsnames]{xcolor}
\definecolor{niceblue}{HTML}{236899}
\newcommand{\rev}[1]{{\color{niceblue} #1}}
\usepackage{hyperref}
\hypersetup{
    colorlinks=true,
    linkcolor=black,
    filecolor=black,
    urlcolor=niceblue,
    citecolor=niceblue,
    linkbordercolor = white
}
\usepackage{longtable}
\usepackage{amsbsy}
\usepackage{epsfig}
\usepackage{bm}
\usepackage{xspace}
\usepackage{color}
\usepackage{lineno}
\usepackage{ragged2e}
\usepackage{comment}

% Linux Libertine:
\usepackage{textcomp}
\usepackage[sb]{libertine}
\usepackage[varqu,varl]{inconsolata}% sans serif typewriter
\usepackage[libertine,bigdelims,vvarbb]{newtxmath} % bb from STIX
\usepackage[cal=boondoxo]{mathalfa} % mathcal
\useosf % osf for texb, not math
\usepackage[supstfm=libertinesups,%
  supscaled=1.2,%
  raised=-.13em]{superiors}
\usepackage{setspace}

\makeatletter
\let\LN@align\align
\let\LN@endalign\endalign
\renewcommand{\align}{\linenomath\LN@align}
\renewcommand{\endalign}{\LN@endalign\endlinenomath}
\let\LN@gather\gather
\let\LN@endgather\endgather
\renewcommand{\gather}{\linenomath\LN@gather}
\renewcommand{\endgather}{\LN@endgather\endlinenomath}
\makeatother

\widowpenalty10000
\clubpenalty10000

\definecolor{darkgrey}{HTML}{A9A9A9}
\renewcommand\linenumberfont{\normalfont\bfseries\small\color{darkgrey}}
\modulolinenumbers[2]

\usepackage{booktabs}
\usepackage[round]{natbib}
\bibliographystyle{icesjms}
\bibpunct{(}{)}{,}{a}{}{,}
\usepackage{authblk}
\usepackage{pdflscape}
\usepackage{setspace}

\newcommand{\pe}[1]{{\color{blue}#1}}
\newcommand{\sa}[1]{{\color{red}#1}}
\newcommand{\kg}[1]{{\color{violet}#1}}
% other colours: orange, magenta, sky blue, purple, grey

\newcommand*{\TitleFont}{%
      \usefont{\encodingdefault}{\rmdefault}{b}{n}%
      \fontsize{13}{15}%
      \selectfont}

% \newcommand{\R}[1]{\label{#1}\linelabel{#1}}
% \newcommand{\lr}[1]{page~\pageref{#1}, line~\lineref{#1}}
% \newcommand{\lr}[1]{line~\lineref{#1}}

\date{}

\title{Impacts on population indices if scientific surveys are excluded from marine protected areas}

% Tentative order (alphabetical after Philina right now) ...
\author[1,2]{Sean C. Anderson}
\author[1]{Philina A. English}
\author[3]{Katie S.P. Gale}
\author[1,4]{Dana R. Haggarty}
\author[5]{Carolyn K. Robb}
\author[3,6]{Emily M. Rubidge}
\author[3,7]{Patrick L. Thompson}

\affil[1]{Pacific Biological Station, Fisheries and Oceans Canada, Nanaimo, BC, Canada}
\affil[2]{Department of Mathematics, Simon Fraser University, Burnaby, BC, Canada}
\affil[3]{Institute of Ocean Sciences, Fisheries and Oceans Canada, Sidney, BC, Canada}
\affil[4]{Department of Biology, University of Victoria, Victoria, BC, Canada}
\affil[5]{Regional Headquarters, Fisheries and Oceans Canada, Vancouver, BC, Canada}
\affil[6]{Department of Forest and Conservation Sciences, University of British Columbia, Vancouver, BC, Canada}
\affil[7]{Department of Zoology, University of British Columbia, Vancouver, BC, Canada}
\affil[*]{corresponding author: sean.anderson@dfo-mpo.gc.ca}

% lost on each survey
\newcommand{\lostHBLL}{0.24}
\newcommand{\lostHS}{0.09}
\newcommand{\lostQCS}{0.09}
\newcommand{\lostWCHG}{0.23}
\newcommand{\nSpp}{41}
\newcommand{\hbllNSpp}{19}
\newcommand{\synNSpp}{35}
 % R output

% \doublespacing
\onehalfspacing
\begin{document}
% \begin{spacing}{1.2}
\maketitle

% \noindent
% Possible target journals: CJFAS, ICES JMS, Fisheries Research, Ecological Applications?

\clearpage
% \linenumbers

\section*{Abstract}

% TODO ICES: 200 words max
% TODO: perhaps could frame this as broader than MPAs (e.g., mostly wind farms in the US) and emphasize the general approach more with the NSB MPA case study
Marine protected areas (MPAs) are increasingly implemented worldwide and typically limit commercial and recreational fishing activities.
However, scientific surveys that impact benthic habitat are sometimes also limited or excluded.
We examine the impacts of excluding scientific surveys on population indices using a proposed MPA network off western Canada, four surveys, and \nSpp\ groundfish species as a case study.
Survey exclusion removed 10--25\% of the area, excluded up to $\approx$50\% of species density, and resulted in minor impacts for most species but considerable impacts for some.
Less precise indices, indices for species more represented in MPAs, and indices for species whose density moved into or out of MPAs were most impacted.
We quantify expected losses of precision, inter-annual accuracy, and trend bias for a percent increase in species density covered by MPAs.
We find that redistributing survey effort outside MPAs would lessen precision loss but with limited benefits to accuracy or trend detection.
While these changes may not qualitatively alter stock assessment outcomes for most species, this is a critical next question.
Losses of power to detect 50\% population declines of up to $\approx$30 percentage points for some species in the most affected survey suggest the impacts may be meaningful in some cases.
If survey restrictions continue expanding, index integrity would further degrade, eventually compromising the ability to manage exploited populations.
Regulating surveys within MPA boundaries therefore requires careful consideration to balance MPA objectives with the need for reliable data.

% to precision, inter-annual accur
% indices with up to \maxPrecisionLost \% reductions in precision, accuracy loss (median absolute relative error) of up to \maxMARE\, and trends deviating by up to $\sim$40\% per decade.

% ---the Northern Shelf Bioregion Marine Protected Area Network---
% would have impacted the precision, accuracy, and trend bias of population indices for \nSpp\ benthic fish species.

% For species with at least 15\% of their density within the MPAs, precision reduced by \precisionWCHGshrunk \% in the most impacted trawl survey and \precisionHBLLshrunk \% in a hook-and-line survey, on average.
% Precision reduced by \covPrecisionGeo\ per 1\% of species density excluded by the survey.
% The coefficient of variation, median absolute relative error increased by \covMareGeo,


% Maximum word count for abstracts: 200 words

% Longer draft:
%% Scientific surveys provide critical data for the assessment of population status, including through indices of population biomass or abundance through time.
%% Marine protected areas (MPAs) are increasingly implemented as one of several approaches to meet spatial conservation targets.
%% While MPAs typically limit commercial or recreational activities, scientific surveys that cause impacts to MPA objectives such as benthic habitat conservation may also be restricted.
%% The impacts of such exclusion to the calculation of indices of abundance across a suite of species and survey types is not well quantified.
%% Here, we conduct a retrospective analysis asking to what extent a proposed network of MPAs---the Northern Shelf Bioregion Marine Protected Area Network off the coast of British Columbia, Canada---would have impacted the precision (coefficient to variation; CV), accuracy (median absolute relative error; MARE), and bias of population indices (relative error trend) of \nSpp\ benthic fish species from three bottom trawl and one hook-and-line depth-stratified randomized surveys.
%% Our results show that the spatial restrictions, which range from 10\% to 30\% across surveys, result in impacts to indices ranging from relatively minor to considerable with up to 40\% increases in imprecision, 40\% increases in inaccuracy, and trends in relative error of up to 40\% per decade.
%% For species with at least 15\% of their density within the MPA, indices of abundance would on average increase their CV 25\% in the most impacted trawl survey and 15\% in a hook-and-line survey.
%% Indices that were less precise to start with suffered larger impacts to accuracy.
%% Indices for species with a higher density proportion in MPAs suffered larger losses of precision, accuracy, and trend bias.
%% Species with a changing proportion of density within vs. outside the MPA suffered the most extreme trend biases as the species became more present or hidden from the survey.
%% We further conclude that reduced survey effort overall is responsible for most of the precision loss, but redistributing survey effort outside the MPA would do little to reduce accuracy loss and trend bias.
%% We also find that the spatially contiguous pattern of MPA restrictions, or the particular MPA locations themselves, are more impactful to accuracy and especially trend bias than an equivalent random effort reduction.
%% Our results are generally similar for geostatistical model-based or design-based estimators.
%%
%% Together, our results suggest small to moderate impacts on survey index integrity for most species if these surveys were restricted from this proposed MPA network.
%% TODO FIX THIS
%%
%%
%% The changes noted here may not meaningfully change most stock assessment outcomes, although this is a critical next question.
%% However, if spatial protection and associated restrictions for scientific surveys continue to expand, index integrity would further degrade and eventually compromise the ability to manage exploited populations.
%% % Therefore, we suggest allowing at least some survey effort in closed areas, particularly from less destructive gear; accelerating research on less-destructive survey methods; and ensuring adequate calibration between any historical and new survey methods to maintain index integrity.
%% Therefore, we suggest careful consideration when regulating scientific surveys within MPA boundaries to understand risk tradeoffs between the conservation objectives of the MPAs and the accuracy of indices, and other data products such as sampled ages of fish, needed for the assessment and management of exploited populations.

\noindent

Keywords:
groundfish,
index of abundance,
MPA,
scientific survey,
spatiotemporal,
spatio-temporal,
survey effort reduction





% \clearpage
% \tableofcontents
% \thispagestyle{empty}
% \setcounter{page}{0}
% \clearpage

% ER: FYI Protection Standards to better conserve our oceans (dfo-mpo.gc.ca)
% These minimum standards to MPAs apply to federal MPAs only but may be adopted by other policy tools, and include the following gear restrictions:
% The new protection standard for MPAs prohibit four key industrial activities in all new federal MPAs: oil and gas activities; mining; dumping; and bottom trawling.
% The prohibition on bottom trawling applies to mobile bottom-contact gear used for commercial and recreational purposes, including otter trawls, beam trawls, shrimp trawls, hydraulic clam dredges, and scallop dredges.
% All other activities, including fishing that does not use mobile bottom-contact gear, will continue to be assessed on a case-by-case basis to ensure they do not pose a risk to the conservation objectives of the MPA.
% *Bottom trawling for Indigenous food, social, and ceremonial purposes and for scientific research purposes will be allowed within MPA where it does not pose a significant risk to the MPA's conservation objectives.*
%
% This last point is important because it clearly states that science surveys can continue if no risk to cons. Obj.  of MPA (unlikely for most of our MPAs as benthic features often Conservation Objectives).

\section*{Introduction}

% <!-- Paragraph 1: scientific surveys are important and pervasive -->
Scientific surveys form the backbone of population status assessment for exploited marine species \citep[e.g.,][]{doubleday1981, hilbornwalters1992, gunderson1993}. One of the primary products of such surveys are indices of relative or absolute stock biomass or abundance through time (herein population or stock ``indices''). These indices are vital to most fisheries stock assessment \citep{hilbornwalters1992} and to several other applications such as the assessment of species at risk \citep{iucn2012}. Because of this importance, government agencies spend considerable effort to maintain random, stratified random, or fixed-station sampling designs as well as consistent gear and survey timing so that indices remain proportional to underlying trends in population biomass or abundance.

% <!-- Paragraph 2: globally MPAs are being proposed more and more -->
Globally, regions of the ocean are increasingly set aside as reserves or marine protected areas (MPAs) to protect ecosystems, species groups, fisheries, and culturally and socially valued areas from a range of anthropogenic threats. Motivated in part by the direction of the United Nations Convention on Biological Diversity to protect at least 10\% of marine areas by 2020 (since increased to 30\% by 2030), the area of MPAs worldwide has increased from 0.7\% to $\sim$8\% over the last 22 years \citep{WDPA2022}. Although the level of protection of individual MPAs varies from highly protected to multiuse \citep{hortaecosta2016, grorud-colvert2021}, it is clear that full- or high-protection MPAs (i.e., low or no extractive or destructive activities) have the best conservation outcomes \citep[e.g.,][]{lester2008, sciberras2015, edgar2014, zupan2018}. Commercial bottom trawling is therefore frequently restricted in MPAs. In Canada, for example, excluding industrial bottom trawling has been recommended as a ``Protection Standard'' for all of conservation areas that meet the IUCN definition of an MPA \citep{dfo2022standards}.

% <!-- Paragraph 3: restricted from fishing, but then the question becomes scientific surveys -->
While MPA regulations typically target commercial and recreational fishing activities, a focus on building more effective MPAs has also raised concerns about impacts on benthic habitat from scientific survey gear \citep{field2006, ursgroup2016, saarman2018, benoit2020national}. Although research surveys typically have much lower benthic footprints than commercial fleets---primarily because of their considerably smaller scale \citep{benoit2020national}---hook-and-line survey gear, and in particular trawl gear, damages benthic habitat \citep[e.g.,][]{collie2000a, kaiser2006, hiddink2017}. In Canada, scientific research surveys that pose a risk to conservation objectives may be excluded from MPAs and fisheries closures. In most cases, a permitting process determines if the research activities constitute an exception under the MPA regulations. For example, Rockfish Conservation Areas, a collection of over 160 areas in Pacific Canada aiming to protect sensitive rockfish populations, prohibit several modes of commercial and recreational and fishing as well as trawl or longline extractive research surveys \citep{thornborough2020}.

% <!-- # Paragraph 4: if excluded, this presents a number of possible issues -->
If scientific surveys are excluded from MPAs based on the gear used and negative impacts to benthic conservation objectives, there are several concerns related to loss of precision, inter-annual accuracy, and trend bias in indices derived from these surveys \citep{field2006, rideout2017, saarman2018, benoit2020national, benoit2020gulf, ices2023}.
A reduced number of survey sets is expected to reduce index precision according to statistical sampling theory \citep{cochran1977}, although survey effort may be redirected to the remaining survey domain.
Loss of survey coverage from portions of a population's distribution may also introduce bias if relative abundance inside and outside the closed areas changes because of fish movement or recovery within closed areas \citep[][]{benoit2020national}.
An analyst faced with a restricted survey must furthermore decide whether to ``shrink'' the survey domain by removing the restricted area \citep{rideout2017, benoit2020national} or extrapolate/interpolate from the surveyed region into the restricted area \citep{benoit2020national}.
This decision may have implications for precision, accuracy, and bias and therefore scientific advice to management.

% <!-- # Paragraph 5: what we do here -->
Here, we develop a general approach for evaluating the impacts of restricting scientific surveys on population indices through a retrospective analysis of survey data.
We apply this approach to a proposed MPA network off western Canada---the Northern Shelf Bioregion MPA Network \citep[][Fig.~\ref{fig:map}]{dfo2022networkactionplan}---as a case study.
We use spatiotemporal-model-based and traditional design-based methods to derive indices of relative biomass or abundance for four surveys (three using trawl gear and one using long-line gear) and \nSpp\ groundfish species and recalculate those indices as if scientific surveys had been restricted from areas proposed in the MPA network.
We evaluate changes in index precision, accuracy, trend bias, and power to detect decline, and evaluate correlates of these changes.

\section*{Methods}

We first describe a general approach for evaluating the impact of proposed survey restriction on indices of abundance or biomass with existing data.
We then describe the specifics of how we applied this approach to four groundfish surveys within a proposed MPA network off the coast of western Canada.

\subsection*{Calculating survey indices}

We derived both design-based and spatiotemporal-model-based (herein ``model-based'') indices of abundance.
% In many regions, design-based estimators have traditionally been used \citep{schnute2000} REFs; however, geostatistical model-based indices are increasingly used in stock assessments \citep[e.g.,][]{thorson2015a} REFs.
Our results, however, focus on model-based indices because of a general move towards this approach, because the approach tends to have better precision than design-based indices \citep{shelton2014, thorson2015a}, and because model-based approaches are likely better suited to dealing with unanticipated changes to survey design \citep{ices2023}.

Our design-based random stratified estimation follows \citet[][p.~91]{cochran1977} and takes the mean abundance within each depth stratum and the mean of those means.
We used a bootstrap procedure to quantify variance of this estimator \citep{schnute2000}.
The bootstrap procedure performed the same index calculation after sampling with replacement by depth stratum 1000 times and calculating the standard deviation of the resulting means \citep{schnute2000}.
We also calculated a classic stratified random sampling variance \citep[][p.~95]{cochran1977} although the results were similar to the bootstrap procedure and so we do not include them here.

We derived model-based indices of relative abundance or biomass using a spatiotemporal modelling approach with spatial and spatiotemporal Gaussian Markov random fields describing latent spatial and spatiotemporal processes \citep[e.g.,][]{shelton2014, thorson2015a, anderson2022}.
Depending on the species and survey, these models were configured as either delta models \citep{aitchison1955} with a logit link for a binomial model of encounter probability  and log link for a gamma model of positive catch weight \citep{schnute2003} or as \citet{tweedie1984} observation models with a log link for catch weight.
We included an offset term \citep[][p.~206]{mccullagh1989} for log area swept for the trawl models and log hook count for the long-line models.
Following a common model configuration for index standardization \citep[e.g.,][]{thorson2015a, thorson2019a}, our models include an independent intercept each year, a fixed spatial random field, and independent spatiotemporal random fields.
We reduced model complexity as needed based on diagnostics to reflect an analyst aiming for adequate model convergence and parsimony (Table~\ref{tab:model-configs}, Supporting Material).
We fit our models with the sdmTMB \textsf{R} package \citep{anderson2022}, which implements the models through TMB \citep{kristensen2016} with input matrices generated through \textsf{R}-INLA \citep{lindgren2015}.
Additional modelling details are available in the Supporting Material.

% \subsection*{Comparing indices to the status quo}

We calculated a series of indices. The first two form the main analysis and the third is an additional option we focus on mainly in supplemental figures.

\begin{enumerate}

\item Status quo: estimate the index based on all available data and the full survey domain. This becomes our comparison index when assessing changes to precision, accuracy, or bias.

\item Shrunk: estimate the index based on survey observations outside the proposed MPA network and a reduced survey domain that excludes the MPA network. We expect this to be the most likely outcome if survey effort is restricted although in the future some survey effort may be redirected over the shrunk survey domain.

\item Extrapolated: using observations outside the MPA network,  estimate the area-weighted index of abundance for the full status quo survey domain. This option could take advantage of nearby data, spatial correlation, and possible covariates to predict density within the MPA and retain the full survey domain.
\end{enumerate}

\subsection*{Metrics of precision, accuracy, and trend bias}

We compared performance measures of the restricted shrunk or extrapolated indices to the status quo indices that use the full set of historical survey observations:

\begin{enumerate}
  \item Precision: We calculated the annual coefficient of variation $\mathrm{CV}_{t,s}$ for year $t$ and survey~$s$ as $\sqrt{\exp(\mathrm{SE}_{t,s}^2) - 1}$, where $\mathrm{SE}$ is the standard error on the index as calculated with the generalized delta method in TMB \citep{kristensen2016} for the model-based index. We then calculated the proportional change in $\mathrm{CV}_{t,s}$ between each restricted index and the status quo index. This represents a standardized measure of precision loss.

    \item Accuracy: We calculated the annual relative error (RE) as $(\hat{\theta}_t - \theta_t) / \theta_t$, where $\hat{\theta}_t$ represents the index estimate from the restricted model in year $t$ and $\theta_t$ represents the index estimate from the status quo survey. We first centered $\hat{\theta}_t$ and $\theta_t$ by dividing them by their respective geometric means across years to avoid including consistent bias (an index from a shrunk survey domain should be lower than a full-survey domain index) and because these surveys are considered relative indices of abundance/biomass with an estimated catchability parameter that is estimated in stock assessments. We then calculated the absolute RE. The median absolute RE (MARE) represents a measure of accuracy loss---the expected magnitude of relative error in a typical year.

    \item Trend bias: We determined the trend in the relative error described above by fitting a linear regression to RE by year and reporting the slope in units of RE per decade.

\end{enumerate}

\subsection*{Correlates of survey-restriction impacts}

We examined three potential predictors of changes in precision, accuracy, and trend bias:
(1) the CV of the status quo index, (2) the biomass or abundance proportion of a species inside the MPA network, and (3) trends in proportion of a species inside vs.\ outside the MPA.
For the first two, we fit generalized linear models (GLMs) with gamma observation error and log links with log-transformed predictor variables to identify patterns.
This implies that a proportional change in the predictor corresponds to a proportional change in the response.
For any species with slight proportional decreases in precision, we set these values to small positive values (0.01) for the purposes of this GLM because the log link requires positive data.
For the third, we compared the slope of relative error per decade against slope of density proportion within the MPA per decade for the most impacted survey, SYN WCHG and fit a linear regression.
% We did not fit these models to evaluate statistical significance (noting that we ignore uncertainty in the predictor and response variables) but to characterize the relationships.
Because these are proposed MPAs, they should not have had any causal effect on historical biomass or abundance.

\subsection*{Examining the role of effort reduction}

If surveys are restricted from MPAs, a likely scenario is that survey effort would be redistributed throughout the remaining survey domain.
We therefore examined two additional scenarios to assess the contribution of effort reduction itself vs.\ the specific spatially blocked nature imposed by the MPAs.
(1) Down-sampled: do not remove survey sets from the MPAs explicitly, but remove an equivalent level of survey effort by randomly removing survey sets from the entire survey domain, with equal numbers removed from each survey depth stratum.
(2) Up-sampled: redistribute the lost survey effort by assigning an equivalent number of new survey sets within each stratum and generate plausible observations for these new locations by simulating new observations from the status quo spatiotemporal model (Supporting Methods).
We created five random versions of each resampled dataset, calculated the same indices as the main analysis, and calculated the same metrics of precision, accuracy, and trend bias.

% TODO: debatably, this should come a bit earlier:

\subsection*{Power analysis of the reduced ability to detect decline}

We performed a power analysis to identify the reduced ability to detect an exponential decline in population trend under the MPA-restricted scenario compared to the status quo.
We focused our analysis on species with at least 15\% of their biomass or abundance density within the MPA network and only for the surveys that were most impacted in precision and accuracy.

We used the spatiotemporal models fit to the full dataset (Supporting Material) and simulated new datasets with the fitted spatial random fields, new simulated spatiotemporal random fields, and observation error based on the fitted distribution.
In the simulation, we replaced the year factor effects with exponential declines in biomass or abundance density as a linear function in log link space.
For models simulated from a delta-gamma distribution, we held the detection probability constant.
We chose rates of decline that would be meaningful and that illustrated contrast between the two sampling scenarios: 50\% decline for species with average status quo CVs of $>$ 0.25 and 30\% decline for species with CVs $\le$ 0.25.

We then fit spatiotemporal models (omitting spatiotemporal random effects for simplicity, consistency, and estimation speed) with the same observation family and tested whether the upper 95\% CI on the slope parameter representing the effect of year was less than zero.
We repeated the simulation and estimation 120 times to calculate power: the proportion of times that the estimation models correctly identified a negative population trend.

\subsection*{Application to an MPA network off the West Coast of Canada}

\subsubsection*{Proposed Northern Shelf Bioregion MPA network}

We applied the above approach to a specific MPA network and set of surveys off the West Coast of Canada.
The MPA network planning area, the Northern Shelf Bioregion (NSB), is approximately 102,000~km$^2$ and the proposed MPA network encompasses about 30\% of this area (30,493~km$^2$).
% TODO: this still correct?
The network process has been co-led by 15 partner First Nations, the province of British Columbia, and the Canadian federal government (Parks Canada, Environment and Climate Change Canada, and Fisheries and Oceans Canada). 
% CR: Parks Canada, EEEC and DFO? Might be worth mentioning, just because they represent a diversity of spaces (not just fisheries management).
% TODO: ER: is this correct?
Following a systematic conservation planning approach \citep{margules2000}, its design was guided by national and regional goals \citep{canada2011, canada2014} and centered on meeting network conservation objectives \citep{dfo2022networkactionplan}.
Of the six goals of the network, Goal 1 is of primary importance: ``protect and maintain marine biodiversity, ecological representation and special natural features.''
Additional goals and objectives include the protection of traditional use and cultural heritage, contributing to the protection of fisheries resources and their habitat, and providing opportunities for scientific research, education, and awareness \citep{canada2014}.
The proposed network is built off existing areas, where nearly 50\% of the draft scenario footprint already exists as some type of spatial protection (existing provincial or federal MPAs or fisheries closures in the form of Rockfish Conservation Areas) leaving $\sim$50\% of the network as new areas identified through the planning process \citep{dfo2022networkactionplan}.
The proposed NSB MPA network was announced by the planning partners in February 2023 with plans for iterative implementation; the first set of new MPAs are expected to be established by 2025.

The NSB MPA network proposes `Category 1' and `Category 2' zones.
Category 1 zones are proposed for implementation by 2025 and Category 2 zones by 2030 \citep{dfo2022networkactionplan}. 
In our analysis, we assume both zones are restricted from bottom trawl and long-line research surveys. 
From the existing Gwaii Haanas National Marine Conservation Area Reserve and Haida Heritage Site, we excluded trawl surveys from all ``Strict Protection'' zones, while only excluding the long-line survey from ``Strict Protection'' zones that overlap with Rockfish Conservation Areas. We do not exclude survey activity from remaining existing MPA zones where possible changes to activities are uncertain at this time \citep{dfo2022networkactionplan}.

% CR: Similar comment for this paragraph, to the casual reader (e.g. not from BC), these details may not mean much.
% SA: I think it's important to keep; was important to ER etc. plus this it the 'case-study' section
Among the Category 1 and 2 zones, our analysis represents a worst-case scenario for survey indices, since it is unlikely that all survey effort would be restricted from all areas.
In reality, access to sites for research may vary depending on the area's designation and the potential for the surveys to contribute to monitoring the MPA.
We also assume all Category 1 and 2 zones will ultimately be implemented as MPAs.

\subsubsection*{Scientific surveys}

We analyzed data from four research surveys including the three synoptic bottom trawl surveys \citep[][Fig.~\ref{fig:map}]{sinclair2003syn, anderson2019synopsis}: Queen Charlotte Sound \citep[SYN QCS,][]{williams2018synqcs}, Hecate Strait \citep[SYN HS,][]{wyeth2018synhs}, and West Coast Haida Gwaii \citep[SYN WCHG,][]{williams2018synwchg}. We omitted two related surveys off the west coast of Vancouver Island because they are not affected by the proposed MPA network. The included surveys cover the continental shelf and upper slope off the northern part of the west coast of Canada (Fig.~\ref{fig:map}) and primarily target soft-bottom habitat. They have been conducted since 2003, are typically run in alternating years, and follow a random depth-stratified design with sampling units of 2 km $\times$ 2 km blocks with consistent gear and fishing protocols \citep{sinclair2003syn}. The unit of measurement for the synoptic trawl surveys is fish biomass per area swept.

The fourth survey we analyzed was the Outside Northern Hard Bottom Longline survey (HBLL OUT N), which covers a similar geographic region to the synoptic trawl surveys, but instead focuses on hard bottom habitat \citep{doherty2019hbllout} (Fig.~\ref{fig:map}). This survey has generally been run in alternating years since 2006, is conducted by Fisheries and Oceans Canada in collaboration with the Pacific Halibut Management Association, and also follows a random depth-stratified design with sampling units of 2 km $\times$ 2 km blocks \citep{doherty2019hbllout, anderson2019synopsis}. The unit of measurement for the HBLL OUT N survey is numbers of fish per assumed area of catchability based on 8 ft (2.438 m) hook spacing and an assumed catch radius of 30 ft (9.144 m).

Based on a threshold of at least 5\% of survey sets detecting a species, we selected \nSpp\ species of benthic fishes including rockfish (\textit{Sebastes}), flatfish (\textit{Pleuronectiformes}), Pacific cod (\textit{Gadus macrocephalus}), and several elasmobranch species (Table~\ref{tab:spp-sci}). Of these species, \synNSpp\ met this threshold in the synoptic trawl surveys and \hbllNSpp\ met this threshold in the longline survey.

\section*{Results}

Full exclusion of scientific surveys from the proposed NSB MPA network zones would have resulted in a proportional loss of survey domain of \lostHBLL, \lostWCHG, \lostHS, and \lostQCS\ across the HBLL, SYN WCHG, SYN HS, and SYN QCS surveys, respectively (Fig.~\ref{fig:map}, Fig.~\ref{fig:strata}).
% If existing MPAs also excluded survey effort, \lostHBLLall, \lostWCHGall, and \lostQCSHSall\ of the survey domain would be excluded.
The impact of the proposed MPAs on indices was variable across species and surveys when looking at the time series (Fig.~\ref{fig:timeseries}, Fig.~\ref{fig:ts-hbll}--\ref{fig:ts-qcs}).
Overall trends were often similar between the status quo and shrunk scenarios, although the shrunk scenario had larger confidence intervals and increased interannual variability.
The increased interannual variability led to smoother generalized additive model fits in some cases---flattening some patterns that were apparent with the status quo indices (e.g.,~Fig.~\ref{fig:timeseries}).

\subsection*{Metrics of precision, accuracy, and trend bias}

Restricting a survey nearly universally reduced precision (Fig.~\ref{fig:dotplot}a--d) and accuracy (Fig.~\ref{fig:dotplot}e--h) with respect to the status quo index and sometimes resulted in trend bias (Fig.~\ref{fig:dotplot}i--l).
Impacts on precision, accuracy, and trend bias were strongest in the SYN WCHG trawl survey (Fig.~\ref{fig:dotplot}a, e, i) followed by the HBLL OUT N survey (Fig.~\ref{fig:dotplot}b, f, j).
These two surveys contained species with the highest proportion of density within the proposed MPA network with proportions up to $\approx$0.5 (Fig.~\ref{fig:dotplot}).
For species with at least 15\% of their biomass or abundance density within the MPA, the average CV increase was \precisionWCHGshrunk\% in SYN WCHG and \precisionHBLLshrunk\% in HBLL OUT N (Fig.~\ref{fig:dotplot}a, b).
Accuracy loss, MARE---reflecting proportional error in a typical bad year---ranged up to \canaryrockfishSYNWCHGslopere\ (Canary Rockfish) in SYN WCHG and \southernrocksoleHBLLslopere\ (Southern Rock Sole) in HBLL OUT N with an average for species with at least 15\% density within the MPA of \accuracyWCHGshrunk\ and \accuracyHBLLshrunk\ in these surveys, respectively (Fig.~\ref{fig:dotplot}e, f).

Trends in RE ranged up to \canaryrockfishSYNWCHGslopere\ per decade for canary rockfish in SYN WCHG, meaning the index diverged from the status quo index by up to \canaryrockfishSYNWCHGsloperePerc\% per decade (Fig.~\ref{fig:dotplot}i).
In the HBLL OUT N, trends in RE per decade were largest for North Pacific Spiny Dogfish at \northpacificspinydogfishSYNWCHGslopere\ per decade.
Effects within the SYN QCS, where only four species had $>$15\% density within the MPA, were the smallest (Fig.~\ref{fig:dotplot}d, h, l).

\subsection*{Correlates of survey-restriction impacts}

There were strong relationships between some covariates and our metrics of precision, accuracy, and trend bias for the shrunk-survey-domain indices (Figs.~\ref{fig:covariates} and ~\ref{fig:slopes}).
Indices with larger CVs tended to have greater losses of accuracy with survey restriction than indices starting with smaller CVs (Fig.~\ref{fig:covariates}a).
Across all surveys and species, this was at a rate of \covCvmaregeostat\ increase in MARE per 1\% increase in status quo index CV.
Species with a greater proportion of their stock within the MPAs suffered larger losses of accuracy and precision (Fig.~\ref{fig:covariates}b, c).
Across all surveys, a 1\% increase in proportion of a species within the MPA network resulted in an increase in MARE of \covMareGeo\ and an increase in CV of \covPrecisionGeo.

There was a nearly one-to-one negative relationship between the slope of density proportion within the MPAs per decade and the slope of relative error per decade (regression slope: \REslopeRegress, Fig.~\ref{fig:slopes}).
Visualizations of the most extreme cases illustrate these trends in proportion (Fig.~\ref{fig:slopes}, right column)---Canary Rockfish, North Pacific Spiny Dogfish, Redstripe Rockfish, and Petrale Sole were increasingly found in higher density proportions outside the proposed MPA zones.
In other words, proportionally, these species were increasingly seen by the survey resulting in index time series in the restricted scenario that trended up more positively (or not down negatively enough) compared to the status quo full-survey-domain index.

Extrapolating the species distributions into the closed areas instead of restricting the survey domain reduced precision with little or no overall benefit to accuracy or trend on average (Fig.~\ref{fig:dotplot-extrapolate}).
Calculating the indices with design-based methods instead of spatiotemporal models resulted in similar qualitative conclusions (Fig.~\ref{fig:metrics-design-geo}, \ref{fig:covariates-design}) but with slightly larger CVs in most cases (Fig.~\ref{fig:design-model-cv}).

\subsection*{Examining the role of effort reduction}

Through two resampling scenarios, we evaluated the contribution of reduced sampling effort to the above results (Fig.~\ref{fig:sampling}).
In the scenario where we removed an equivalent effort randomly from each strata, we observed similar precision loss to the MPA exclusion scenario but with improved confidence interval coverage, accuracy loss, and trend bias (Fig.~\ref{fig:sampling}, orange vs.\ black).
In the scenario where we up-sampled survey effort to maintain an equivalent total effort distributed across the now restricted surveyed domain,
precision loss was almost entirely ameliorated but at the expense of poorer confidence interval coverage and accuracy (Fig.~\ref{fig:sampling}, blue vs.\ black).
% TODO: CR: The implication that you make here also implies that the MPAs are biased towards some species? Did you test if it was wrong about the trend as well? In the next paragraph you look at the precision relative to the trend, but if you design calculate/estimate the model inside the MPA areas, do you get the same trend as in the "open" areas? This speaks to an analysis from Cordue 2006 who looked at untrawlable areas and concluded if the trends were going in opposite directions, that's bad, but if the trends go in the same direction its OK, cause you just fit a lower Q, precision notwithstanding (I'm paraphrasing and trying to remember the exact result). This comment you can choose to address or not, I'm mostly just curious. I guess to answer this you would fit the model in the closed area, fit it in the open area and see if the trend is the same. I guess you kind of get this from teh RE on the trend over decades.
This implies that the redistributed additional survey effort resulted in an index that was increasingly precise about the ``wrong'' values (with correct defined as the full status quo index).

\subsection*{Power analysis of the reduced ability to detect decline}

Reductions in power to detect population declines were generally stronger in the SYN WCHG trawl survey than the HBLL OUT N longline survey (Fig.~\ref{fig:power}).
As examples, in the SYN WCHG trawl survey, the probability of detecting a 50\% decline for Redstripe Rockfish declined from 0.64 to 0.38.
Power to detect a 30\% decline for Redbanded Rockfish declined from 0.82 to 0.55.
% See 05-power.R 'NUMBERS FOR PAPER' to grab these hardcoded values
Reduction in power was generally 10 percentage points or smaller for the HBLL OUT N survey with the largest effects seen for Copper Rockfish and Tiger Rockfish.
Some species, such as Lingcod, saw minimal effects on power for both surveys.

\section*{Discussion}

We undertook an analysis of historical survey data to assess expected impacts on population indices if scientific surveys are excluded from a proposed MPA network.
Our analysis used the NSB MPA network on the west coast of Canada as a case study, but our approach is applicable to decisions about scientific survey exclusion worldwide.
Our results documented a range of impacts on precision, accuracy, and bias in trend estimation.
Species with less precise indices, species with a higher relative density in the restricted area, and species whose distribution changed with respect to the excluded area were most impacted.
For many species, the impacts were small to moderate and are unlikely to have strong effects on science advice.
However, in other cases, impacts on precision and accuracy had notable effects on the power to detect population decline.
Furthermore, biases in population trend---driven largely by changes in distribution inside vs.\ outside the excluded areas---could impact stock assessment results and estimates of decline for conservation agencies.
Analyses such as ours are an important part of assembling the evidence needed when making decisions about restricting or allowing scientific surveys within a previously established survey domain.
Here, we discuss the magnitude of these effects, why some effects may have a greater impact than others, consequences beyond stock assessment, caveats to our approach, and make recommendations based on our analyses.

\subsection*{Are these impacts enough to make a difference?}

Our results quantified effects on precision and accuracy of survey indices.
Are these effects large enough to impact stock assessment?
In QCS and HS, the impacts of these survey exclusions are probably small enough that they are unlikely to have major consequences for stock assessment outcomes.
However, there may be impacts for some species within the SYN WCHG and the HBLL OUT N survey.
Our analysis found notable reductions in the power to detect 30\% and 50\% declines, which hint that these effects may meaningfully impact stock assessment estimates.
Ultimately, the impact on stock assessments will require further analysis---either with retrospective analyses degrading stock assessment input data, similar to that performed here, or through simulation.
A complete analysis would include closed-loop simulation \citep{smith1999, punt2016} under restricted survey scenarios to evaluate how these effects might impact the probability of achieving conservation and fishery objectives.

% Paragraph about how surveys tend to be designed to be right at the border of
% utility to balance precision with costs and resources
Some of the less-precise survey indices may already be pushing the bounds of utility; however, they remain the most important source of information for assessing commercially fished stocks.
In the surveys examined here, precision for many stocks is already above the targeted \citep{sinclair2003syn} 0.2 CV and stocks with higher CVs were proportionally more affected by lost survey coverage.
Surveys are often designed to have just enough precision for their intended use due to the substantial resources needed to run them \citep{hilbornwalters1992}.
Furthermore, survey indices are often assumed to suffer from additional process error (e.g., changes to catchability due to different fishing captains) and so additional error is often added when entering the index into an assessment \citep{pennington1994, francis2003}. 
This further reduces an already imprecise index's ability to inform an assessment model.
There is no hard cut off of precision and accuracy in terms of impact on stock assessment---the impacts occur on a continuous spectrum.
Even if the specific effects identified here only have moderate impacts on stock assessment, the issue could be ``death by a 1000 cuts'' if survey restriction continues to grow in the future \citep{benoit2020national}.

We hypothesize that trend bias (trend in relative error) will have the greatest impact on stock assessment, recovery monitoring, and conservation agency assessment.
For some species, this trend bias was considerable.
For example, North Pacific Spiny Dogfish, Canary Rockfish, Redstripe Rockfish, and Petrale Sole had increases in relative error of $\approx$20--30\% per decade.
Index precision is sometimes ignored in simple stock assessments \citep{free2020} or adjusted through re-weighting schemes \citep{francis2011} and many index-based management procedures ignore precision in their calculations all together \citep{carruthers2016}.
A bias in trend, however, will directly impact index-based management procedures and affect the ``data'' to which stock assessments compare their calculations in the likelihood.
Additionally, if MPAs operate as hoped, the proportion of a species occurring inside the MPAs will initially increase over time, which would be expected to cause the restricted indices to underestimate the true overall species abundance resulting in a negative trend in relative error.
Therefore, trend bias may be even more present with real MPAs compared to our retrospective analysis where future MPAs had no causal mechanism to affect the past.
Fortunately, this potential for underestimation of abundance is consistent with the precautionary principle.

We evaluated impacts on indices of abundance if surveys are excluded from closed areas; however, survey exclusion would also impact biological samples such as age and length composition data.
The loss of such sampling within closed areas might be expected to result in larger impacts on stock assessment than degraded indices of abundance themselves (\citeauthor{yin2004} \citeyear{yin2004} but see \citeauthor{chen2003} \citeyear{chen2003}).
Composition data, and especially age composition data, are important inputs to modern integrated age structured stock assessments and provide high-quality information regarding recruitment, gear selectivity, growth, natural mortality, and relative depletion \citep{magnusson2007, ono2015}.
Furthermore, age and length samples are notoriously clustered \citep{hilbornwalters1992, francis2011}---fish of a similar age tend to be caught together---and removing spatial blocks from the survey domain could have a disproportionate impact on age composition data if the MPA happened to protect some non-random portion of the population (e.g., a nursery area for young fish, or a refuge or feeding area for older fish).
Finally, while spatiotemporal models are increasingly used for overall biomass or abundance density modelling (as done in this paper) and could use spatial information, covariates, or other surveys to alleviate the impact to indices, such approaches are less well-established for composition data \citep[but see][]{thorsonhaltuch2018}.

A reduction in the spatial domain of surveys and the degradation of index precision, accuracy, and trend estimation will impact our ability to make valid inference for other purposes beyond stock assessment.
Surveys support conservation planning analysis of biodiversity patterns and habitat suitability \citep[e.g.,][]{rubidge2016, thompson2022meps} and understanding species' potential responses to climate change \citep[e.g.,][]{english2021, thompson2023warming}.
Indices are also used by conservation agencies such as IUCN globally \citep{iucn2012} and COSEWIC in Canada \citep{cosewicTable2} to assess rates of decline and assign population status; the results of our power analysis to detect decline is particularly relevant to this use.
Surveys could also be used to monitor recovery effects of MPAs themselves against long historical baselines of expected spatial and temporal patterns and levels of observation error.
In fact, the surveys examined here were used to inform the MPA network design whose impacts we are evaluating.

\subsection*{Caveats and next steps}

There are several caveats to our analysis.
First, we performed a retrospective analysis on a historical dataset.
A simulation approach \citep[e.g.,][]{schnute2003, regular2020} could explore additional dimensions of the problem (e.g., systematically varying observation error, sampling intensity, or species distribution characteristics), derive potentially more generalizable results, and have a known truth for comparison instead of using the status quo index as a reference.
Conversely, using a historical dataset ``bakes'' in the ``messiness'' of reality (e.g., poorly understood outliers, clustering, movement, and spatially varying abundance trends) and makes the results pertinent to real stocks and real surveys where management decisions are made.
% TODO: next is kind of duplicated elsewhere?
Second, we suspect that surveys covering larger areas (e.g., Gulf of Alaska) might fare better under survey restriction when using the spatiotemporal modelling approach compared to the smaller surveys examined here (especially West Coast Haida Gwaii).
Compared to other surveys, West Coast Haida Gwaii spatiotemporal indices were more affected by survey exclusion than their design-based equivalent.
This might be because the survey is on the verge of lacking sufficient spatial data to support these spatiotemporal models.
Third, we found that indices were most impacted for species that had a higher relative abundance or biomass within the MPAs.
These species might also be expected to be most protected by the MPAs and it is possible this protection could offset some of the risk incurred by poorer survey coverage.
However, the degree of protection will depend on several factors including juvenile and adult species movement \citep[e.g.,][]{gerber2003, gruss2011} and larval dispersal distance \citep[e.g.,][]{botsford2003, planes2009} and knowledge about whether the protection is working will be challenging to establish without continued surveying within the MPAs.

% TODO: come back to---duplicating elsewhere
% Our analysis of impacts on survey indices raises several important follow-on questions.
% We suspect that losing age and length composition samples may have a stronger effect on stock assessment estimates than the degraded survey indices examined here---such composition data are known to be important (REFs).
% Furthermore, added bias or imprecision on stock assessment estimates such as measures of status with respect to reference points and fishing intensity are only impactful if they affect science advice and the probability of achieving conservation and fishery objectives.
% Management strategy evaluation would be a key tool for addressing such questions.

% TODO: this paragraph is a bit rough still:
Whether or not omitting a spatial portion of a stock from the survey domain impacts stock assessment outcomes will depend largely on the degree to which fish move \citep{field2006}.
At one extreme, if the stock is stationary and has little larval exchange, the portion of the population in the MPA could be considered an independent stock and shrinking the survey domain would have little impact on assessment outcomes.
However, if a species freely moves into and out of the MPA---and especially if the stock does not maintain a static distribution---the survey will no longer represent an index that is proportional to abundance.
We expect most of the groundfish examined here fall towards the latter scenario.
Movement tracking, oceanographic studies of larval dispersal, and continued monitoring within the MPA could help help inform these scenarios.

\subsection*{Recommendations}

If surveys are excluded from a portion of the original domain, our results suggest that survey effort should likely be redistributed, but this is not a panacea.
This will recover most of the lost precision, which is primarily lost due to reduced sample size.
However, the increased sampling outside the MPAs could make accuracy (with respect to the status quo index) worse, and do little to remedy biased perceptions of trend.
Our results suggest that impacts on trend can be caused by the spatially contiguous and specific spatial region of survey loss combined with changes in species density within vs.\ outside the MPAs \citep[a form of spatially varying population trends,][]{barnett2021a}---no amount of additional sampling outside of the MPA can solve this problem.
In these cases, increased sampling merely makes us more precise about the ``wrong'' index because of the missing part of the population.

Should we use design- or model-based indices when faced with survey restrictions?
Overall, we found similar effects for design- and model-based indices.
However, we see several advantages of the spatiotemporal model approach in this context.
First, the models produced more precise indices in the status quo scenario \citep[sensu][]{thorson2015a}, and even though proportionally they were more impacted by restriction than the design-based index, they remained more precise, on average.
% TODO: some duplication with elsewhere here:
Second, spatiotemporal model indexes are likely to be less impacted for surveys covering larger spatial areas or with denser spatial coverage than some surveys considered here.
We suspect the spatiotemporal models were already struggling for WCHG where large spatial chunks were then removed.
If spatial information, covariate effects, or even spatially varying covariate effects \citep{hastie1993, barnett2021a, thorson2023svc} could be estimated well, we suspect a spatiotemporal model may be able to improve accuracy compared to omitting the MPA from the survey domain prediction grid by extrapolating into the closed area, but at the expense of precision.
% TODO: some duplication with the previous point here:
Third, the model-based approach affords the opportunity to leverage other data sources such as covariates \citep{johnson2019} and other surveys \citep[e.g.,][]{gruss2019, webster2020, monnahan2021} to enhance indices.
The approach provides a mechanism to make use of historical data in areas that later restrict surveys rather than discarding all historical data the does not match new stratum definitions.
Fourth, the model-based approach provides a statistical framework to make inference about the benefits of the MPAs through causal inference \citep[e.g.,][]{pearl2016}, assuming some sampling within the MPAs is maintained.

Our results suggest that we should consider maintaining at least some survey effort within MPAs---using minimally destructive gear where possible.
Longline surveys will tend to have less benthic impacts than bottom trawl surveys; however, longline gear (1) will tend to select for larger bait-attracted individuals than caught in trawl gear and (2) tend to not sample the same species as trawl surveys.
Most groundfish species are well indexed by either longline or bottom trawl gear, and few species are indexed well by both \citep[e.g.,][]{anderson2019synopsis}.
Longline surveys are not, therefore, a direct substitute for trawl surveys in most cases.
% \citep{field2006}
We suggest accelerating research on less-destructive survey methods, such as underwater camera systems
% CR: ROVs are actually pretty uncommon in the survey literature. There's 1 ongoing ROV survey in the GOA for yelloweye rockfish I can think of, but most attempts at using cameras to survey fish use less expensive options (BRUVs, towed cameras, stationary cameras, etc). and of course manned submersibles, so a more general image based survey method statement might be better
\citep[e.g.,][]{trenkel2004, rooper2012, bryan2023} and eDNA approaches \citep[e.g.,][]{rourke2022, he2023}, but acknowledge these approaches are not a complete substitute for trawl surveys \citep{benoit2020national}.
% CR: Kind of disagree with this statement. These approaches will never fully replace trawl surveys in that they won't be able to collect specimen data. However, they are not that far from replacing trawl surveys in terms of estimating biomass. Especially acoustic surveys have gotten really good at replacing trawl surveys (or at least reducing trawling). I've got a bit better citation for this, I'll send it along
% SA: agreed, good points, text has been modified
In addition to changes in catchability and size selectivity, most alternative approaches lack an opportunity to collect biological samples; information on age and maturity are critical inputs to stock assessment \citep[e.g.,][]{magnusson2007}.
Fish lengths may be more readily collected, such as visually with image-based surveys, but due to asymptotic growth, length provides little information on population age structure for a large portion of most groundfish lifespans.
% For example, a XX cm long TODO Rockfish could be XX or XX years old.
If new survey methods are adopted, it will be critical to calibrate new methods with historical ones to ensure unbroken time series \citep{field2006}.
As an alternative, or supplement to alternative methods, some level of reduced survey intensity with extractive gear within closed areas might be chosen to balance benthic protection objectives with monitoring of other species groups.

% \textbf{Paragraph: Consider the risk tradeoff explicitly}

Eliminating scientific surveys inside protected areas shifts risk from those associated with localized disturbance of benthic habitats and organisms (including corals and sponges) onto a suite of commercially exploited species (groundfish here) by reducing the ability to effectively monitor their populations.
Many of these these commercially exploited species regularly move into and out of the MPAs and some are at low fractions of their unfished biomass.
How to balance this tradeoff is a policy decision, but we think it is important to remember that this tradeoff exists.
Decisions about how to manage this tradeoff should be rooted in analysis, such as what we have explored here, as well as estimates of survey gear recurrence time \citep{benoit2020national} and simulation modelling identifying impacts on probability of achieving conservation and fishery objectives.
% We suggest allowing at least some survey effort in closed areas, particularly from less destructive gear, accelerating research on less destructive survey methods, and ensuring sufficient calibration between any historical and new survey methods to maintain index integrity.

Although we applied our approach to a proposed MPA network, our approach is equally applicable to other forms of survey restriction including wind farms \citep{hare2022}, oil and gas, funding cuts, or other logistical restraints \citep{ices2023}.
Structured decision making benefits from an understanding of consequence tradeoffs incurred by potential decisions \citep{gregory2012}.
Our approach is a structured way to assess expected impacts from unanticipated reductions in survey coverage using historical data.
% TODO: experimental? thinking adaptive management style experiments I guess
We see it as important step in analyzing the problem---complimented by simulation-based and experimental analyses.
% MPAs are an opportunity to monitor and learn about effectiveness of these strategies... but need to make sure they don't hamper our ability to 
As long as we as a society let aquatic species be exploited commercially, we have an obligation to maintain the integrity of the long-term data streams that are critical to managing the health our marine ecosystems.
% Conservation efforts to support marine biodiversity should not, ironically, hamper our ability to assess marine species. 

% \textbf{Paragraph: Do analyses such as these}
% We have unveiled a structured way to think about the problem
% Widely applicable to other surveys
% and impacts other than MPAs (e.g., wind/NOAA)
% Such be a first step in any decision making

% Longline surveys tend to have less benthic impacts than bottom trawl surveys, but longline doesn't sample all species well (e.g. POP) and selects for larger bait attracted individuals - in very few cases a substitute! \citep{field2006}
% accelerating research on less-destructive survey methods (ROV, eDNA, acoustic, ...), again, long way from being substitute \citep{field2006}
% but not just indices - also composition data and other life-history info (growth, maturity) - need some physical samples
% ensuring adequate calibration between historical and new less-destructive - includes length/age selectivity in addition to overall catchability \citep{field2006}
% It's possible a reduced survey intensity with destructive gear within closed areas could best balance benthic protection objectives with monitoring of other species groups.

% \citep{mcgilliard2015}:
% e.g. of using spatial stock assessment to reduce impacts of MPAs on assessments
% "assessing populations as a single stock with use of fishery catch-rate data and without accounting for the NTMR results in severe underestimation of biomass for two of the movement pattern"

% \citep{field2006}:
% "Although non-lethal means of estimating abundance (such as direct observation surveys from submersibles or towed vehicles) offer an alternative way to track abun- dance and size structure, particularly in areas difficult or impossible to sample with trawl or other extractive survey gear (Richards 1986; O’Connell and Carlile 1993; Jagielo et al. 2003)"
% Other non-lethal survey methods in- clude egg or larval abundance (or production) surveys (Mangel and Smith 1990; Moser et al. 2000; MacCall 2003; Ralston et al. 2003) and hydroacoustic surveys (Stanley et al. 2000; Helser et al. 2006),
%  However, any new indices of abundance based on non-lethal methods cannot replace historical time series without a period of overlap and calibration.

% Punt and Methot (2004) conducted a quantitative evaluation of the effects of implementing large-scale MPAs on the performance of stock assessment methods. Using the Management Strategy Evalua- tion approach,5 they found that while the negative impacts of MPAs on assessments were substantial if MPA and non-MPA data were aggregated, the impacts were minor when spatially structured models were used

% Bias and errors increased with movement rate, particularly when movement in- creased with age, because the models treated each area separately; performance was less sensitive to assumptions about larval dispersal and density dependence. Not surprisingly, a lack of fishery- independent survey data for the population inside the MPA led to considerably degraded model performance.

\section*{Acknowledgements}

We thank the many individuals who have contributed to collecting the survey data on which this manuscript is based.
We thank C.N. Rooper whose comments on an earlier draft greatly improved this manuscript.
% TODO there will be more

\section*{Data Availability Statement}

The survey data used in this study are openly available through \href{open.canada.ca}{open.canada.ca} for the \href{https://open.canada.ca/data/en/dataset/524fde54-1d93-4d22-bb83-df542780a719}{HBLL OUT N longline survey} and the \href{https://open.canada.ca/data/en/dataset/a278d1af-d567-4964-a109-ae1e84cbd24a}{synoptic trawl surveys}.
Code for this analysis is available at \href{https://github.com/pbs-assess/gfmpa}{github.com/pbs-assess/gfmpa} and will be archived with \href{https://zenodo.org/}{Zenodo} on publication.

\bibliography{refs}

% \end{spacing}

\clearpage

% \section*{Tables}

% \clearpage

\section*{Figures}

\begin{figure}[htb]
  \centering
  \includegraphics[width=6in]{figs/fig1.pdf}
  \caption{Map of proposed Marine Protected Areas (MPAs) within the Northern Shelf Bioregion (NSB) MPA network that overlap with included surveys. Left panel shows the survey domains as shaded grey in 2 $\times$ 2 km grid cells and the MPAs coloured by zone category. Right panel shows all survey sampling locations between 2003 and 2022 as coloured dots (colour indicating survey) and with the MPAs coloured in black.}
  \label{fig:map}
\end{figure}


% \begin{figure}[htb]
%     \centering
%     \includegraphics[width=4in]{figs/map-existing-restrictions.png}
%     \caption{Map of current groundfish survey footprints within the Northern Shelf Bioregion. Survey colours ordered by increasing proportion covered by proposed MPAs (not illustrated) and dots represent all survey sampling locations between 2003 and 2021. Red polygons indicate existing restricted areas.}
%     \label{fig:map}
% \end{figure}

% 2. time series
\begin{figure}[htb]
    \centering
    \includegraphics[width=\textwidth]{figs/index-geo-restricted-highlights}
    \caption{Indices of relative abundance for selected species with variable patterns in abundance and responses to restrictions. Dots and lines represent means and 95\% CIs. Colours lines and points represent the status quo index (with colour indicating the survey) and grey represents the same time series estimated assuming survey sets within the NSB MPA network had been excluded and the survey domain shrunk. All time series have been centered to have a geometric mean of one and the upper vertical axis is limited to four times the maximum mean estimate per panel (affects Petrale Sole in 2007). See Figures \ref{fig:ts-hbll}, \ref{fig:ts-wchg}, \ref{fig:ts-hs}, and \ref{fig:ts-qcs} for all time series. ``Rougheye/Blackspotted'' refers to the Rougheye and Blackspotted Rockfish complex. Generalized additive models with gamma error and log links are shown to highlight trends.}
    \label{fig:timeseries}
\end{figure}

% 3. CV/ MARE/ trend in RE
\begin{figure}[htb]
    \centering
    \includegraphics[width=0.92\textwidth]{figs/metrics-dotplot-main}
    \caption{Variation across species in the loss of precision (a--d), accuracy (e--h), and trend bias in relative error (RE: i--l) for indices of abundance resulting from retrospective survey restriction in the shrunk-survey-domain scenario. Dots represent medians and lines represent interquartile ranges across annual index values (a--h) and the mean and 95\% CI on the estimated slope from a linear model of RE per decade (i--l). Vertical grey line indicates no change from the restriction and dashed coloured vertical lines indicate medians within a category (across all species, including those not shown). Surveys are sorted with those most impacted spatially by MPA restrictions at the top. Species are sorted from top to bottom within surveys by proportion of predicted biomass or abundance within the NSB MPA regions (shown in parentheses) and species are omitted if their proportions with the NSB MPA regions are less than 0.1 (1/26 for SYN WCHG, 0/19 for HBLL OUT N, 22/32 for SYN HS, 25/29 for SYN QCS; all species are shown in Fig.~\ref{fig:covariates}). See Fig.~\ref{fig:metrics-design-geo} for a version with design-based estimators and Fig.~\ref{fig:cv-abs} for absolute CVs. }
    \label{fig:dotplot}
\end{figure}

\begin{figure}[htb]
  \centering
  \includegraphics[width=\textwidth]{figs/metrics-cross-plot1.pdf}
  \caption{Relationships between status quo index uncertainty (CV), proportion of a species' biomass occurring within the NSB MPA network, and measures of accuracy and precision loss for indices of abundance in the shrunk-survey-domain scenario. Dots represent median values across years for individual species-survey combinations. Lines and shaded ribbons represent means and 95\% CIs from generalized linear models with gamma error and log links. Colours represent individual surveys and the black line represents all surveys combined.}
  \label{fig:covariates}
\end{figure}

\clearpage

\begin{figure}[htb]
    \centering
    \includegraphics[width=4.8in]{figs/slopes-wchg2}
    \caption{Slope in relative error (i.e., how ``wrong'' the estimated index trend is) as a function of trend in the proportion of a species' density that is located within the MPA network (i.e., whether a species is increasing in density inside vs.\ outside the MPA). Shown here is the SYN WCHG survey where the strongest trends occurred. Species to the right of zero are increasingly hidden by the MPAs; species to the left are increasingly found outside the MPAs. Panels on the right illustrate the most extreme cases of trends in density proportion within the MPAs with GAMs fit for visualization. Colours in left and right panels match. Solid sloped line and ribbon represent a regression mean and 95\% CI; dotted sloped line represents a 1:1 relationship.}
    \label{fig:slopes}
\end{figure}

\clearpage

\begin{figure}[htb]
    \centering
    \includegraphics[width=\textwidth]{figs/sampled-dotplot-comparison2}
    \caption{Evaluating the contribution of reduced sampling effort. Random down-sampled is a scenario where we did not remove survey sets from the MPAs explicitly, but randomly (within strata) removed an equivalent level of survey effort from the entire survey domain. This evaluates the role of the particular location and blocked nature of the MPA removals vs.\ survey effort loss alone. Random-up-sampled and shrunk (blue) is a scenario where an equivalent number of survey sets lost in the MPA network are simulated in the remaining survey domain to represent a redistribution of the same effort. Restricted and shrunk is the standard MPA-restricted scenario illustrated elsewhere in the analysis. Panels a, c, and d show the same metrics as in Fig.~\ref{fig:dotplot}. Panel b shows the proportion of replicates across species (and five random seeds) in which the 50\% confidence intervals include the status quo index mean.}
    \label{fig:sampling}
\end{figure}

\clearpage

\begin{figure}[htb]
    \centering
    \includegraphics[width=3.4in]{figs/power-june6-eps-null-spatial}
    \caption{Statistical power to detect a simulated population decline of a given magnitude over the span of the survey (16 [SYN WCHG] or 15 [HBLL OUT N] years). Arrows begin at power under the status quo scenario and end at power under the scenario where surveys are restricted from the MPA network. We explore two levels of decline depending on the survey CV: 50\% decline for surveys with CVs $>$ 0.25 and 30\% decline for surveys with CVs $\le$ 0.25. We fit models with spatial random fields and a linear fixed effect of year (in log link space) on abundance or biomass density. We then defined power as the frequency that the upper 95\% CI on year fixed effect was correctly below zero. Species are arranged by length of arrows within decline categories. Numbers in parentheses represent average status quo index CVs.}
    \label{fig:power}
\end{figure}

% % 6. trend in MPA vs. trend in RE
% \begin{figure}[htb]
%     \centering
%     \includegraphics[width=0.8\textwidth]{figs/explore-all-slopes2.pdf}
%     \caption{Relationship between the proportion of a species' biomass occurring within MPAs and the change in relative error per decade (RE trend) resulting from `extrapolated' indices of abundance from all four surveys. Selected species are labelled.}
%     %  TODO: Maybe update fig 2 to include Silvergray and others highlighted for QCS and HBLL here?
%     \label{fig:slopes}
% \end{figure}

% - HBLL N and QCS

\renewcommand{\thefigure}{S\arabic{figure}}
\renewcommand{\thetable}{S\arabic{table}}
\setcounter{figure}{0}
\setcounter{table}{0}
\setcounter{section}{0}
\setcounter{subsection}{0}
\setcounter{subsubsection}{0}

\clearpage

\appendix
% \internallinenumbers

\addtocontents{toc}{\protect\setcounter{tocdepth}{0}}

\setcounter{secnumdepth}{0} % Removes section numbers

\onehalfspacing
\linenumbers
\resetlinenumber
\setcounter{page}{1}
\setcounter{equation}{0}
\nolinenumbers

\begin{Center}
\section*{Supporting Information}
\end{Center}

\section*{Supporting Methods}

\subsection*{Spatiotemporal modelling}

We employed the stochastic partial differential equation (SPDE) approximation to Gaussian random fields for computational efficiency and fit our models with the R package sdmTMB \citep{anderson2022}, which uses INLA \citep{lindgren2015} for SPDE ``mesh'' construction, TMB \citep{kristensen2016} to calculate the marginal log likelihood, and the \texttt{nlminb} non-linear optimizer in R \citep{r2023} to find the parameter values that maximize the marginal log likelihood.
We constructed the meshes with vertices at least 10 km apart.
We ensured the optimization of all models was consistent with convergence by checking that the Hessian was positive definite, the maximum gradient with respect to fixed effects was $<$ 0.001, and all random field marginal standard deviations were $\ge$ 0.01.
From our fitted models, we calculated area-weighted indices of abundance by predicting on a 2 $\times$ 2 km grid covering the survey domain, summing the predicted biomass or abundance each year \citep{thorson2015a}, and applying a generic bias-adjustment given the non-linear transformation of the random effects \citep{thorson2016bias}.

Our models took the general form:
\begin{align}
\mathbb{E}[y_{\boldsymbol{s},t}] &= \mu_{\boldsymbol{s},t},\\
\mu_{\boldsymbol{s},t} &=
f^{-1} \left(
\alpha_g +
O_{\boldsymbol{s},t} +
\omega_{\boldsymbol{s}} +
\epsilon_{\boldsymbol{s},t} \right),
\end{align}
\noindent
where
$y_{\boldsymbol{s},t}$ represents the response data at point $\boldsymbol{s}$ and time $t$ (catch weight or count),
$\mu$ represents the mean,
$f$ represents a link function (here logit or log) and $f^{-1}$ represents its inverse,
$\alpha_{g}$ represents independent intercepts for each year,
$O_{\boldsymbol{s},t}$ represents an offset (log transformed area swept or hook
count),
$\omega_{\boldsymbol{s}}$ represents a spatial random field with Mat\'ern covariance [$\omega_{\boldsymbol{s}} \sim \mathrm{MVN}(\boldsymbol{0},\boldsymbol{\Sigma}_\omega)$],
and $\epsilon_{\boldsymbol{s},t}$ represents spatiotemporal random fields with Mat\'ern covariance [$\epsilon_{\boldsymbol{s},t} \sim \mathrm{MVN}(\boldsymbol{0},\boldsymbol{\Sigma}_{\epsilon})$].
The delta-gamma models have two components: a binomial component with a logit link modelling species encounter vs.\ non-encounter and a gamma component with a log link modelling species catch (biomass or count) if a species was observed in a given set.
The Tweedie models use Tweedie observation error with a log link.

We place penalized complexity priors \citep{fuglstad2019} on the Mat\'ern covariance parameters. This weakly constrains the parameters to reasonable values and will tend to favour simpler random fields (less `wiggly' fields with smaller standard deviations and larger ranges) if the data are not informative about spatial correlation. We set priors on both the spatial and spatiotemporal marginal standard deviation as $P(\sigma > 5) = 0.05$ and a prior on the range parameter shared across any spatial and spatiotemporal fields within a model component as $P(\rho < 20\ \mathrm{km}) = 0.05$.

We fit a sequence of models to emulate what an applied modeller might do if manually fitting models to these data sets and choosing a model that is best able to represent variation in the data while achieving satisfactory convergence diagnostics (Table~\ref{tab:model-configs}). For each survey and species combination and in the restricted survey scenario, the first model is fit; if that model fails convergence checks (positive-definite Hessian matrix, maximum gradient with respect to fixed effects $<$ 0.001, all random field standard deviations $\ge$ 0.01) then the subsequent model is run and so on. We start by fitting models to the ``restricted'' survey data sets since these have least data and therefore tend to have the most convergence issues. We then record the chosen model and use this same model when fitting the full data set for comparison (Fig.~\ref{fig:metrics-by-model}).

\clearpage

\section*{Supporting Tables}

\input{figs/spp-table.tex}

\clearpage

\begin{table}[htpb]
\caption{Sequence of model configurations attempted. All configurations have
spatial random fields. The final model chosen for each species-survey combination is shown in Fig.~\ref{fig:metrics-by-model}.}
\centering
\begin{tabular}{lllll}
\toprule
Model run & Family & Spatiotemporal fields  \\
\midrule
1 & Delta-gamma & Included in binomial and gamma components  \\
2 & Delta-gamma & Included in gamma model only \\
3 & Tweedie & Included  \\
4 & Delta-gamma & None \\
\bottomrule
\end{tabular}
\label{tab:model-configs}
\end{table}

\clearpage

\section*{Supporting Figures}

\begin{figure}[htb]
    \centering
    \includegraphics[width=\textwidth]{figs/grid-strata-restricted.pdf}
    \caption{Depth strata (coloured cells) and their overlap with the NSB MPA network (black-outlined cells).}
    \label{fig:strata}
\end{figure}

\clearpage

\begin{figure}[htb]
    \centering
    \includegraphics[width=\textwidth]{figs/ts-hbll.pdf}
    \caption{Outside Northern Hard Bottom Longline survey (HBLL OUT N) indices. Dots represent means and line segments represent 95\% confidence intervals. Colour represents the full ``Status quo'' survey and grey represents the survey if it had been restricted by the NSB MPAs. All time series have been centered to have a geometric mean of one to eliminate changes to the scale of the indices. Species are ordered by decreasing proportion of density within the NSB MPAs from top to bottom. Curved lines represent fitted generalized additive models with gamma error distributions and log links to visually highlight differences in longterm trend.}
    \label{fig:ts-hbll}
\end{figure}

\clearpage

\begin{figure}[htb]
    \centering
    \includegraphics[width=\textwidth]{figs/ts-wchg.pdf}
    \caption{West Coast Haida Gwaii (SYN WCHG) indices. Caption otherwise the same as for
    Fig.~\ref{fig:ts-hbll}.}
    \label{fig:ts-wchg}
\end{figure}

\clearpage

\begin{figure}[htb]
    \centering
    \includegraphics[width=\textwidth]{figs/ts-hs.pdf}
    \caption{Hecate
      Strait (SYN HS) indices. Caption otherwise the same as for Fig.~\ref{fig:ts-hbll}.}
    \label{fig:ts-hs}
\end{figure}

\clearpage

\begin{figure}[htb]
    \centering
    \includegraphics[width=\textwidth]{figs/ts-qcs.pdf}
    \caption{Queen Charlotte Sound (SYN QCS) indices. Caption otherwise the same as for Fig.~\ref{fig:ts-hbll}.}
    \label{fig:ts-qcs}
\end{figure}

\clearpage

% \begin{figure}[htb]
%     \centering
%     \includegraphics[width=\textwidth]{figs/ts-qcs-hs.pdf}
%     \caption{Queen Charlotte Sound (SYN QCS) and Hecate
%       Strait (SYN HS) indices. These two neighbouring
%   surveys that are conducted one after the other with the same gear have been
% combined into one index. Caption otherwise the same as for Fig.~\ref{fig:ts-hbll}.}
%     \label{fig:ts-qcs-hs}
% \end{figure}

% \clearpage

\begin{figure}[htb]
    \centering
    \includegraphics[width=0.90\textwidth]{figs/abs-cv}
    \caption{Coefficient of variation (CV) of design-based and spatiotemporal-model-based indices of abundance. Whereas Fig.~\ref{fig:dotplot}a--c shows percent increase in CV if the survey was restricted from MPAs, this figure shows the absolute CV. Dots show medians and line segments show the interquartile range (across years). Dashed horizontal lines show the overall median within a method and survey for the illustrated species. Only species with at least 15\% of their biomass or abundance density within the MPA region are shown (numbers in parentheses beside species names). Vertical black lines show the stated target of CV = 0.2.}
    \label{fig:cv-abs}
\end{figure}


\begin{figure}[htb]
    \centering
    \includegraphics[width=0.99\textwidth]{figs/metrics-dotplot-extrapolate}
    \caption{Same as Fig.~\ref{fig:dotplot} but adding a version where the species density is extrapolated into the restricted areas instead of reducing the survey domain and showing all species.}
    \label{fig:dotplot-extrapolate}
\end{figure}

\begin{figure}[htb]
    \centering
    \includegraphics[width=0.99\textwidth]{figs/metrics-dotplot-design-geo}
    \caption{Same as Fig.~\ref{fig:dotplot} but adding a version where the indices of abundance were calculated as random stratified design-based indices.}
    \label{fig:metrics-design-geo}
\end{figure}

\clearpage

% \begin{figure}[htb]
%     \centering
%     \includegraphics[width=0.8\textwidth]{figs/metrics-dotplot-main-design}
%     \caption{Same as Fig.~\ref{fig:dotplot} but with design-based indices for all species where a design-based estimator could be calculated. Unlike Fig.~\ref{fig:metrics-design-geo}, this version includes some rarer species for which geostatistical models could not be fit.}
%     \label{fig:metrics-design}
% \end{figure}

% \clearpage

\begin{figure}[htb]
    \centering
    \includegraphics[width=5in]{figs/prop-mpa-vs-metrics-design2.pdf}
    \caption{Precision loss, accuracy loss, and trend bias as a function of proportion of stock in the MPA for the scenario where survey effort is restricted in the MPAs and the survey domain is shrunk. This is similar to Fig.~\ref{fig:covariates} but (1) adds design-based indices (left column) and (2) extends the left side of the x-axis to include species with low proportions within the MPA. Some increases in CV that were slightly negative in the top row have been set to 0.01 so a gamma GLM with a log link could be fit (dots along the zero line). Colours represent individual surveys and the black lines represent all surveys combined.}
    \label{fig:covariates-design}
\end{figure}

\clearpage

\begin{figure}[htb]
    \centering
    \includegraphics[width=\textwidth]{figs/geostat-vs-design2}
    \caption{Model- vs.\ design-based CVs. Dashed line shows the one-to-one line. The spatiotemporal-model-based indices tend to have higher precision (lower CVs) than the design-based bootstrap estimators. Panel a shows the status quo scenario and panel b shows the scenario where survey effort has been excluded from the MPA network.}
    \label{fig:design-model-cv}
\end{figure}

\clearpage

\begin{figure}[htb]
    \centering
    \includegraphics[width=0.95\textwidth]{figs/metrics-dotplot-by-model.pdf}
    \caption{And illustration of which spatiotemporal model was selected for each species-survey combination. `Off' means no spatiotemporal random field was used for that component. For example, ``Off, IID'' means the binomial component had no spatiotemporal random effects but the gamma component has spatiotemporal random effects that were independent each year (but with a common standard deviation). All models included spatial random effects and independent means for each year.}
    \label{fig:metrics-by-model}
\end{figure}

\clearpage

\end{document}


At long last, a summary of the upsampling/downsampling. It was really tricky to get this right. Upsampling usually removes most of the CV loss, but that's not entirely helpful because CI coverage (compared to the full survey) often also drops, MARE is about the same, and RE trend is helped a little bit or not at all. Random downsampling of an equivalent effort by strata has a similar impact on CV to the MPA restrictions (except WCHG where it's not quite as bad), has similar or better CI coverage to the restricted scenario, a bit less impact on accuracy, and most notably has less impact on RE trend (2-3 times less for the most impacted surveys).

25 words or less:

First attempt:

Restricting surveys within MPAs has impacts on precision, accuracy, and bias in population trend. Accuracy impacts on strongest for already less-precise indices, accuracy and especially precision are most affected by proportion of stock within the MPA, and trend bias is stongly impacted by a species moving into or out of an MPA over time.

Half length:

Restricting surveys within MPAs impacts index precision, accuracy, and trend bias. Less precise indices, species more within MPAs, and movement in/out of MPAs increases impacts.


"
. Francis et al. (2003) analyzed a large number of
M.N. Maunder, A.E. Punt / Fisheries Research 70 (2004) 141–159 155 CPUE and research survey data sets and found that the coefficient of variation for the combined effects of observation error and annual variation in catchability was ∼0.15–0.2.
""

offshore wind energy US:
https://media.fisheries.noaa.gov/2022-03/NOAA%20Fisheries-and-BOEM-Federal-Survey-Mitigation_Strategy_DRAFT_508.pdf

designed to have about 20\% CV? Schnute and Haig CJFAS sim
and Sinclair Res Doc - precision criteria was <= 20\%

"Fishing surveys are one of the most powerful tools in fisheries stock assessment work (Gulland 1988)."

Sinclair:
"Starr and Schwarz (2000) describe a calculation of the biomass change that would be detectable from a given CV. For example, a 20\% CV gives the ability to detect a relative biomass change of 50\% between two observations with 95\% confidence (assuming an underlying log-normal distribution; Figure 1). Similarly, a 30\% CV can detect a 70\% relative change."


Starr and Schwarz:
"If the main consideration is to detect a decline (or rise) in relative biomass, a 20\% RE implies that it would take an ~50\% relative change to be 95\% confident that the survey will detect that decline between any two observations (Figure 7). The corresponding calculation for a change in relative biomass with a 30\% RE is ~70\% relative change."

"Note that this calculation assumes that the RE is a reasonable estimate of the total error in the mean biomass index. Because the RE is only an estimate of the sampling error, then this estimate of a detectable decline is probably a minimum. Therefore, it is proposed that the initial target RE for any species to be monitored by this survey should be no greater than 20\%."



% TODO could discussion stuff from this paragraph

% Precision was reduced by \precisionWCHGshrunk\%, \precisionHBLLshrunk\%, and \precisionQCSHSshrunk\%, on average, and inaccuracy was introduced with average MARE values of \accuracyWCHGshrunk, \accuracyHBLLshrunk, and \accuracyQCSHSshrunk\ across the SYN WCHG, HBLL, and SYN QCS/SYN HS surveys.
% However, these averages hide considerable variation across species---some species were more heavily affected.
% For example, the CV of the Shortspine Thornyhead (\textit{Sebastolobus alascanus}) index in the HBLL survey was increased by \shortspinethornyheadHBLLprecision\% and the CV of the Rougheye/Blackspotted Rockfish complex (\emph{Sebastes aleutianus/melanostictus}) index in the SYN WCHG survey increased by \rougheyeblackspottedrockfishWCHGprecision\%.
% Accuracy (MARE) losses were as high as \chinarockfishHBLLaccuracy\ for China Rockfish (\textit{Sebastes nebulosus}) in the HBLL survey where there was also a trend in relative error of \chinarockfishHBLLbias\ per decade.
% English Sole (\textit{Parophrys vetulus}) experienced the strongest trend in relative error at \englishsoleWCHGbias\ per decade.

% Furthermore, extrapolating to the status quo survey domain resulted in a greater loss of precision than shrinking the survey domain (Fig.~\ref{fig:dotplot}a,d) with a minor improvement in accuracy on average compared to the shrunk survey (Fig.~\ref{fig:dotplot}b,e) (TODO STAT TINY STILL THERE?)
% and a minor reduction in temporal bias (Fig.~\ref{fig:dotplot}c,f; blue dots tend to be farther from 0 than orange dots).
% Restricting a survey introduced temporal bias for some species and surveys with no typical positive or negative trend (Figs~\ref{fig:dotplot}c,f; \ref{fig:re-hbll}, \ref{fig:re-qcs}, \ref{fig:re-hs}, \ref{fig:re-wchg}).
%
% There were several correlates of changes to precision, accuracy, and bias in restricted survey indices compared to the status quo. Survey-species combinations with higher CVs in the status quo index tended to experience greater losses of accuracy and a greater magnitude of trend bias compared to survey-species combinations with lower status quo CVs (Fig.~\ref{fig:cvstatusquo}). Additionally, survey-species combinations with higher proportions of biomass or abundance inside the restricted area tended to experience greater losses of precision and accuracy and a greater magnitude of trend bias compared to survey-species combinations with lower proportions (Fig.~\ref{fig:propmpa}).
%
% Change through time in the proportion of a species' biomass or abundance that was hidden inside proposed restricted areas tended to predict the direction of bias in the restricted indices (Figs~\ref{fig:slopes}, \ref{fig:slopes-by-survey}). Species that were increasingly `hidden' in the proposed closed areas (right half of Fig.~\ref{fig:slopes}) tended to have a negative trend in relative error (lower-right quadrant in Fig.~\ref{fig:slopes}). Conversely, species that increased in biomass or abundance outside the proposed restricted areas (left half of Fig.~\ref{fig:slopes}) tended to have a positive trend in relative error (upper-left quadrant in Fig.~\ref{fig:slopes}).
